\documentclass{article}
\usepackage[spanish]{babel}
\usepackage[utf8]{inputenc}
\usepackage{listings}
\usepackage[left=3cm,right=3cm,top=2cm,bottom=2cm]{geometry}
\usepackage[document]{ragged2e}
\usepackage{color}
\usepackage{textcomp}

\begin{document}

\title{\Huge EC - Práctica 1}
\author{\Large Javier Gálvez Obispo}
\date{\large\today}
\maketitle

\section{Sesión de depuración saludo.s}

  \begin{flushleft}
    \textbf{1. ¿Qué contiene EDX tras ejecutar \textit{mov longsaludo, \$edx}?
    ¿Para qué necesitamos esa instrucción, o ese valor? Responder no sólo el valor correcto
    (en decimal y hex) sino también el significado del mismo (¿de dónde sale?)
    Comprobar que se corresponden los valores hexadecimal y decimal mostrados en la ventana
    Status\textrightarrow Registers } \break

    El registro EDX contiene el valor de \textit{longsaludo} tras la ejecución de \textit{mov longsaludo, \%edx},
    siendo en este caso 0x1C su valor en hexadecimal, 28 en decimal. El valor que almacena \textit{longsaludo}
    es el número de posiciones que ocupa \textit{saludo}, es decir, el número de bytes a escribir.
  \end{flushleft}

  \begin{flushleft}
    \textbf{2. ¿Qué contiene ECX tras ejecutar \textit{mov \$saludo, \%ecx}?
    Indica el valor en hexadecimal, y el significado del mismo.
    Realizar un dibujo a escala de la memoria del programa, indicando dónde empieza
    el programa (\_start, .text), dónde empieza saludo (.data), y dónde está el tope
     de pila (\%esp).} \break

    ECX contiene la dirección de la variable \textit{saludo}. Su valor en hexadecimal es 0x8049097,
    que es la dirección de memoria donde está el contenido de \textit{saludo}.
  \end{flushleft}

  \begin{flushleft}
    \textbf{3. ¿Qué sucede si se elimina el símbolo de dato inmediato (\$) de la
    instrucción anterior? (\textit{mov saludo, \%ecx}). Realizar la modificación, indicar el
    contenido de ECX en hexadecimal, explicar por qué no es lo mismo en ambos casos.
    Concretar de dónde viene el nuevo valor (obtenido sin usar \$).} \break

    Al eliminar el símbolo \$ movemos el contenido de \textit{saludo} a ECX en vez de su dirección.
    Ahora su valor en hexadecimal es 0x616c6f48 que es la representación en hexadecimal del
    contenido de la variable \textit{saludo}.
  \end{flushleft}

  \begin{flushleft}
    \textbf{4. ¿Cuántas posiciones de memoria ocupa la variable \textit{longsaludo}?
    ¿Y la variable \textit{saludo}? ¿Cuántos bytes ocupa por tanto la sección de datos?
    Comprobar con un volcado Data\textrightarrow Memory mayor que la zona de datos
    antes de hacer Run.} \break

    La variable \textit{saludo} ocupa 28B. \break
    La variable \textit{longsaludo} ocupa 1B. \break
    La sección de datos ocupa 29B.
  \end{flushleft}

  \begin{flushleft}
    \textbf{5. Añadir dos volcados Data \textrightarrow Memory de la variable \textit{longsaludo},
    uno como entero hexadecimal, y otro como 4 bytes hex. Teniendo en cuenta lo mostrado en esos volcados...
    ¿Qué direcciones de memoria ocupa \textit{longsaludo}? ¿Cuál byte está en la primera
    posición, el más o el menos significativo? ¿Los procesadores de la línea x86 usan
    el criterio del extremo mayor (big-endian) o menor (little-endian)? Razonar la respuesta.} \break

    Los valcados los hacemos con las siguientes órdenes:  \break
    (gdb) x /1xb \&longsaludo \break
    0x80490b3:	0x1c \break \break

    (gdb) x /4xb \&longsaludo \break
    0x80490b3:	0x1c	0x00	0x00	0x00 \break

    La dirección de memoria de \textit{longsaludo} es 0x80490b3. \break
    En la primera posición está el byte menos significativo. \break
    Los procesadores x86 utilizan el criterio little-endian.
  \end{flushleft}

  \begin{flushleft}
    \textbf{6. ¿Cuántas posiciones de memoria ocupa la instrucción \textit{mov \$1, \%ebx}?
    ¿Cómo se ha obtenido esa información? Indicar las posiciones concretas en hexadecimal.} \break

    La instrucción \textit{mov \$1, \%ebx} ocupa 5 posiciones de memoria.

    (gdb) disas \break
    Dump of assembler code for function \_start:
  \end{flushleft}

  \begin{lstlisting}
     0x08048074 <+0>:	mov    $0x4,%eax
  => 0x08048079 <+5>:	mov    $0x1,%ebx
     0x0804807e <+10>:	mov    $0x8049097,%ecx
     0x08048083 <+15>:	mov    0x80490b3,%edx
     0x08048089 <+21>:	int    $0x80
     0x0804808b <+23>:	mov    $0x1,%eax
     0x08048090 <+28>:	mov    $0x0,%ebx
     0x08048095 <+33>:	int    $0x80
  \end{lstlisting}

  \begin{flushleft}
    End of assembler dump.
  \end{flushleft}

  \begin{flushleft}
    \textbf{7. ¿Qué sucede si se elimina del programa la primera instrucción \textit{int 0x80}?
    ¿Y si se elimina la segunda? Razonar las respuestas.} \break

    Si eliminamos la primera instrucción \textit{int 0x80} no aparece por pantalla el mensaje. \break
    Si eliminamos la segunda instrucción da fallo de violación de segmento al ejecutar el programa. \break
    Esta instrucción se utiliza para las llamadas del sistema.
  \end{flushleft}

  \begin{flushleft}
    \textbf{8. ¿Cuál es el número de la llamada al sistema READ (en kernel Linux 32bits)?
    ¿De dónde se ha obtenido esa información?} \break \break

    El número de la llamada al sistema READ es 3. \break
    Esta información se ve en /usr/include/asm/unistd\_32.h
  \end{flushleft}

\newpage
\section{Sesión de depuración suma.s}

  \begin{flushleft}
    \textbf{1. ¿Cuál es el contenido de EAX justo antes de ejecutar la instrucción RET,
    para esos componentes de lista concretos? Razonar la respuesta, incluyendo cuánto
    valen 0b10, 0x10, y (.-lista)/4.} \break

    El contenido de EAX es 37 que es resultado de todas las sumas realizadas. \break
    0b10 = 2 \break
    0x10 = 16 \break
    (.-lista)/4 = 9
  \end{flushleft}

  \begin{flushleft}
    \textbf{2. ¿Qué valor en hexadecimal se obtiene en \textit{resultado} si se usa la lista
    de 3 elementos: .int 0xffffffff, 0xffffffff, 0xffffffff? ¿Por qué es diferente
    del que se obtiene haciendo la suma a mano? NOTA: Indicar qué valores va tomando
    EAX en cada iteración del bucle, como los muestra la ventana Status\textrightarrow Registers, en
    hexadecimal y decimal (con signo). Fijarse también en si se van activando los flags
    CF y OF o no tras cada suma. Indicar también qué valor muestra \textit{resultado} si se vuelca
    con Data\textrightarrow Memory como decimal (con signo) o unsigned (sin signo).} \break

    En la primera iteración el valor de EAX es 0xffffffff (-1 en decimal con signo),
    en la segunda 0xfffffffe (-2 en decimal con signo), y en la última, 0xfffffffd
    (-3 en decimal con signo). Como en la suma no hemos controlado el acarreo,
    el resultado con signo será -3, pero sin signo será $2^{8}-3 = 253$.
  \end{flushleft}

  \begin{flushleft}
    \textbf{3. ¿Qué dirección se le ha asignado a la etiqueta \textit{suma?}
    ¿Y a \textit{bucle}? ¿Cómo se ha obtenido esa información?} \break

    La dirección de suma es 0x8048095 \break
    La dirección de bucle es 0x80480a0 \break
    Esta información se obtiene con la orden \textit{p \&etiqueta}
  \end{flushleft}

  \begin{flushleft}
    \textbf{4. ¿Para qué usa el procesador los registros EIP y ESP?} \break

    EIP es el puntero a la dirección de la siguiente instrucción y ESP el puntero que
    apunta al inicio de la pila.
  \end{flushleft}

  \begin{flushleft}
    \textbf{5. ¿Cuál es el valor de ESP antes de ejecutar CALL, y cuál antes de ejecutar RET?
    ¿En cuánto se diferencian ambos valores? ¿Por qué? ¿Cuál de los dos valores de
    ESP apunta a algún dato de interés para nosotros? ¿Cuál es ese dato?} \break

    El valor de ESP antes de ejecutar CALL es 0xFFFFD4A0 y antes de ejecutar RET es 0xFFFFD49C.
    La diferencia entre ambos valores es de 4, esta diferencia es producida porque al ejecutar CALL,
    el tamaño de la pila se modifica porque se guarda en ella la dirección de retorno a la función que
    será necesaria cuando se ejecute RET.
  \end{flushleft}

  \begin{flushleft}
    \textbf{6. ¿Qué registros modifica la instrucción CALL? Explicar por qué necesita
    CALL modificar esos registros.} \break

    CALL modifica los registros EAX, EDX, EIP y ESP. \break
    En nuestro programa modificamos EAX para usarlo como acumulador y EDX para usarlo
    como índice. El uso de EIP y ESP está explicado en la pregunta 4.
  \end{flushleft}

  \begin{flushleft}
    \textbf{7. ¿Qué registros modifica la instrucción RET? Explicar por qué
    necesita RET modificar esos registros.} \break

    La instrucción RET modifica los registros ESP y EIP porque los punteros deberán
    cambiar para que el programa pueda continuar con su ejecución al punto donde
    estaba antes de llamar a la subrutina.
  \end{flushleft}

  \begin{flushleft}
    \textbf{8. Indicar qué valores se introducen en la pila durante la ejecución del
    programa, y en qué direcciones de memoria queda cada uno. Realizar un dibujo de
    la pila con dicha información. NOTA: en los volcados Data\textrightarrow Memory se puede usar
    \$esp para referirse a donde apunta el registro ESP.} \break

    El registro ESP al inicio del programa apunta a 0xffffd0f0 con el valor 0xffffd0f0.
    Al llamar a la subrutina suma apunta a 0xffffd0ec	con valor 0xffffd0ec.
    Después de la instrucción \textit{pop \%edx} cambia a 0xffffd0ec y después de
    la instrucción \textit{ret} se modifica a 0xffffd0f0.
  \end{flushleft}

  \begin{flushleft}
    \textbf{9. ¿Cuántas posiciones de memoria ocupa la instrucción \textit{mov \$0, \%edx}?
    ¿Y la instrucción \textit{inc \%edx}? ¿Cuáles son sus respectivos códigos máquina? Indicar
    cómo se han obtenido. NOTA: en los volcados Data\textrightarrow Memory se puede usar una
    dirección hexadecimal 0x... para indicar la dirección del volcado. Recordar la ventana
    View\textrightarrow Machine Code Window. Recordar también la herramienta objdump.} \break

    La instrucción \textit{mov \$0, \%edx} ocupa 5 posiciones de memoria y su código máquina es: \break
    ba 00 00 00 00

    La instrucción \textit{inc \%edx} ocupa 1 posición de memoria y su código máquina es: \break
    42
  \end{flushleft}

  \begin{flushleft}
    \textbf{10.  ¿Qué ocurriría si se eliminara la instrucción RET? Razonar la
    respuesta. Comprobarlo usando ddd.} \break

    Se produce una violación de segmento.
  \end{flushleft}

\newpage
\section{Suma de N enteros \underline{sin} signo de 32bits}

  \begin{lstlisting}
    .section .data
    lista:    .int 1,1,123,1,1,1,1,1
              .int 1,1,1,1,432,1,1,1
              .int 1,456,1,1,1,5234,1,1
              .int 1,1,1,1,1,1,1,234
    longlista:	.int (.-lista)/4
    resultado:	.quad -1

    .section .text
    _start:  .global _start

      mov   $lista, % ebx
      mov   $0, % edx     # % edx lo utilizamos para el acarreo
      mov   longlista, % ecx
      call  suma

      mov   % eax, resultado
      mov   % edx, resultado+4

      mov   $1, % eax
      mov   $0, % ebx
      int   $0x80

    suma:
      push  % esi        # % esi lo utilizamos como indice
      mov   $0, % eax
      mov   $0, % esi

    bucle:
      add   (% ebx,% esi,4), % eax
      jnc   acarreo      # si no hay acarreo saltamos
      inc   % edx        # en caso contrario incrementamos el acarreo

    acarreo:
      inc   % esi
      cmp   % esi, % ecx
      jne   bucle

      pop   % esi
      ret
  \end{lstlisting}

\newpage
\section{Suma de N enteros \underline{con} signo de 32bits}

  \begin{lstlisting}
    .section .data
    lista:    .int -1,-1,-1,-1,-1,-1,-1,-1
              .int -1,-1,-1,-32,-1,12,-1,-1
              .int -1,-1,-1,42,-1,-1,-1,-1
              .int -1,-1,-1,-1,-1,-1,956,-1
    longlista:	.int (.-lista)/4
    resultado:	.quad -1

    .section .text
    _start:  .global _start

      mov   $lista, % ebx
      mov   $0, % edx     # % edx lo utilizamos para el acarreo
      mov   longlista, % ecx
      call  suma

      mov   % eax, resultado
      mov   % edx, resultado+4

      mov   $1, % eax
      mov   $0, % ebx
      int   $0x80

    suma:
      push % esi          # % esi lo utilizamos como indice
      mov $0, % eax
      mov $0, % esi

    bucle:
      mov (% ebx,% esi,4), % ebp   # Movemos el entero a % ebp
      test % ebp, % ebp            # Comprobamos si es negativo
      js negativo                  # Si es negativo saltamos a "negativo"

    positivo:                   # Para sumar un numero positivo
      add % ebp, % eax             # Sumamos % ebp a % eax
      adc $0, % edx                # Sumamos el acarreo (si hay) a % edx
      jmp siguiente

    negativo:                   # Para sumar un numero negativo
      neg % ebp                    # Valor absoluto de % ebp
      sub % ebp, % eax             # A % eax le restamos % ebp
      sbb $0, % edx                # Restamos el debito a % edx

    siguiente:
      inc % esi
      cmp % esi, % ecx
      jne bucle

      pop   % esi
      ret
  \end{lstlisting}

\newpage
\section{Media de N enteros \underline{con} signo de 32bits}

  \begin{lstlisting}
    .section .data
    lista:    .int -1,-1,-1,-1,-1,-1,-1,-1
              .int -1,-1,-1,-32,-1,12,-1,-1
              .int -1,-1,-1,42,-1,-1,-1,-1
              .int -1,-1,-1,-1,-1,-1,956,-1
    longlista:	.int (.-lista)/4
    resultado:	.quad -1

    .section .text
    _start:  .global _start

      mov   $lista, % ebx
      mov   $0, % edx     # % edx lo utilizamos para el acarreo
      mov   longlista, % ecx
      call  suma

      mov   % eax, resultado
      mov   % edx, resultado+4

      # Esta linea es la unica diferencia con la suma con signo
      idiv  %ecx         # Hacemos la division con signo de EDX:EAX entre % ecx
                         # que contiene el valor de longlista

      mov   $1, % eax
      mov   $0, % ebx
      int   $0x80

    suma:
      push % esi          # % esi lo utilizamos como indice
      mov $0, % eax
      mov $0, % esi

    bucle:
      mov (% ebx,% esi,4), % ebp   # Movemos el entero a % ebp
      test % ebp, % ebp            # Comprobamos si es negativo
      js negativo                  # Si es negativo saltamos a "negativo"

    positivo:                   # Para sumar un numero positivo
      add % ebp, % eax             # Sumamos % ebp a % eax
      adc $0, % edx                # Sumamos el acarreo (si hay) a % edx
      jmp siguiente

    negativo:                   # Para sumar un numero negativo
      neg % ebp                    # Valor absoluto de % ebp
      sub % ebp, % eax             # A % eax le restamos % ebp
      sbb $0, % edx                # Restamos el debito a % edx

    siguiente:
      inc % esi
      cmp % esi, % ecx
      jne bucle

      pop   % esi
      ret
  \end{lstlisting}
\end{document}
