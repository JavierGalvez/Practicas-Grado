\documentclass{article}
\usepackage[spanish]{babel}
\usepackage[utf8]{inputenc}
\usepackage{listings}
\usepackage[left=3cm,right=3cm,top=2cm,bottom=2cm]{geometry}
\usepackage{amsmath}
\usepackage{multirow}
\usepackage{hyperref}
\usepackage[document]{ragged2e}
\usepackage{color}

\hypersetup{
    colorlinks=true,
    linkcolor=blue,
    urlcolor=blue,
}

\definecolor{background}{RGB}{246,246,246}

\lstset{ %
  backgroundcolor=\color{background},
  language=c++,
  directivestyle={\color{black}}
}

\begin{document}

\title{\Huge ED - Reto 1}
\author{\Large Javier Gálvez Obispo}
\date{\large\today}
\maketitle


\begin{flushleft}
{\large\textbf{1. Usando la notación O, determinar la eficiencia de los siguientes segmentos  de  código:}}

\begin{lstlisting}
int n,j; int i=1; int x=0;           |      int n,j; int i=2; int x=0;
do{                                  |      do{
	j=1;                         |      	j=1;
	while (j <= n){              |      	while (j <= i){
		j=j*2;               |      		j=j*2;
		x++;                 |      		x++;
	}                            |      	}
	i++;                         |      	i++;
}while (i<=n); 			     |	    }while (i<=n);
\end{lstlisting}

\vspace{\baselineskip}

En el primer código tenemos:
\begin{itemize}
	\item Varias operaciones del orden O(1) que podemos ignorar.
	\item Un bucle \textit{while}, con eficiencia O($log_{2}(n)$) ya que la variable \textit{j} se duplica en cada iteración.
	\item Un bucle \textit{do-while} que se realiza n veces.
\end{itemize}

Por tanto el resultado es una eficiencia de O($n log_{2}(n)$) ya que al estar anidados
los bucles multiplicamos las eficiencias de ambos.

\vspace{\baselineskip} \vspace{\baselineskip}
El segundo código se diferencia del primero en el bucle \textit{while} que
también depende de \textit{i} para el número de itereaciones, por lo que se realizan $log_{2}(i)$ iteraciones.\\
En este caso no podemos multiplicar las eficiencias de ambos bucles y para calcular la
eficiencia del código debemos tratar al bucle \textit{do-while} como una sumatoria desde 1 hasta \textit{n}:

	 $$\sum\limits_{i=1}^{n}{log_{2}(i)} = log_{2}{\prod\limits_{i=1}^{n}i} = log_{2}(n!)$$

y utilizando las propiedades de los logaritmos llegamos a que la eficiencia del código es $log_{2}(n!)$

\vspace{\baselineskip}
\end{flushleft}

\newpage
\begin{flushleft}
{\large\textbf{2. Para cada función $f(n)$ y cada tiempo \textit{t} de la tabla siguiente, determinar
el mayor tamaño de un problema que puede ser resuelto en un tiempo \textit{t} (suponiendo que el
algoritmo para resolver el problema tarda $f(n)$ microsegundos, es decir, $f(n) \cdot 10^{-6}$ sg).}}

\vspace{\baselineskip}

Tenemos la siguiente expresión:
	$$f(n) = t \cdot 10^{6}$$
y para cada $f(n)$ vamos a despejar \textit{n} para poder calcularlo a partir del tiempo.
\end{flushleft}

\begin{itemize}

	\item $f(n) = log_{2}(n)$
		$$log_{2}(n) = t \cdot 10^{6} \implies n = 2^{t \cdot 10^{6}}$$

	\item $f(n) = n$
		$$n = t \cdot 10^{6}$$

	\item $f(n) = nlog_{2}(n)$ (No hay forma analítica en este caso.)
		$$nlog_{2}(n) = t \cdot 10^{6}$$

	\item $f(n) = n^{3}$
		$$n^{3} = t \cdot 10^{6} \implies n = \sqrt[3]{t \cdot 10^{6}}$$

	\item $f(n) = 2^{n}$
		$$2^{n} = t \cdot 10^{6} \implies n = log_{2}(t \cdot 10^{6})$$

	\item $f(n) = n!$ (No hay forma analítica en este caso.)
		$$n! = t \cdot 10^{6}$$

\end{itemize}
\vspace{\baselineskip}

\begin{flushleft}
Obtenemos la siguiente tabla al sustituir \textit{t} por los valores que nos dicen:
\newline
(Para los cálculos se ha utilizado la página: \href{https://www.wolframalpha.com}{Wolfram Alpha})
\end{flushleft}


\begin{table}[!htbp]
\centering
\label{Tabla}
\begin{tabular}{|c|c|c|c|c|c|}
\hline
\multirow{2}{*}{$f(n)$}& \multicolumn{5}{l|}{\hfil$t$} \\ \cline{2-6}
& 1 sg. & 1 h. & 1 semana & 1 año & 1000 años \\ \hline
$log_{2}n$ &  $9.9 \cdot 10^{301029}$  &  $2^{3600 \cdot 10^{6}}$  &  $2^{6 \cdot 10^{12}}$  &  $2^{3.15\cdot 10^{13}}$  &  $2^{3.15\cdot 10^{16}}$ \\ \hline
$n$ &  $10^{6}$  &  $3600 \cdot 10^{6} $  &  $6.04 \cdot 10^{12}$  &  $3.15 \cdot 10^{13}$  & $3.15 \cdot 10^{16}$  \\ \hline
$nlog_{2}n$	&  $62746$  &  $1.33 \cdot 10^8$  &  $1.77 \cdot 10^{10}$  &  $7.97 \cdot 10^{11}$  &  $6.41 \cdot 10^{14}$ \\ \hline
$n^3$	&  $100$  &  $1532$  &  $8456$  &  $31593$  & $315938$  \\ \hline
$2^n$	&  $19$  &  $31$  & $39$   &  $44$  & $54$  \\ \hline
$n!$	& $9$ &	$12$ & $14$ & $16$ & $18$ \\ \hline

\end{tabular}
\end{table}

\end{document}
